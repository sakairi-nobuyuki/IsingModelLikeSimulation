\documentclass{article}
\usepackage{amsmath,latexsym,bm}

\author{N. Sakairi}
\title{The origin of meta-structural strength of Montmorillonite solution using Ising model like numerical model}

\begin{document}
\maketitle

\section{Introduction}
The author has proposed the existence of yield stress, a part of rheologycal behaviour of visco-elastic solution,
was assumed to be originated so-called edge-face interaction \cite{sposito}\cite{adachi}\cite{sakairi}

In, \cite{sakairi}, rather strong repulsive force between sheet like particles of Montmorillonite was assumed.
This is implying wide range of propulsice force due to electrical double layers are dominative while the salt concentration was lower,


An Ising model like numerical model was proposed to explain how each sheet like Montmorillonite particles arrayed in solution.
From the theoretical point of statistical mechanics with classical manners, equillibrium state of large number of particles 
are usually simulated by Ising model. This model assuming only one degree of freedom, of which values are 1 and -1, with 
location of each particles are equally arranged.
While the interaction potential between particles are given with a coefficient $J$ and relatopnship between $i$ th and $j$ th particles.
Thus the Hamiltonial, $H$, of the system is given by, $H = -J\sum_{i<j} s_is_j$, 
where $\sum_{i<j}$ denoting summation over every possible combinations of particles.

While Boltzmann factor a state is proportional to the probability of the system is in the state.
In the subject system, the orientation of the particle is of interest, due to the assumption of E-F or F-F interaction is dominative.
Therefore, each state of the system can be represented by combinations of particle orientation.
If we assume, $s_i$ to be the orientation of the $i$ th particle in the system, the state of system can be described as a group with the representation of 
 $S = \{s_1, s_2, \cdots, s_n\}$.
For the purpose of simplicity, $s_i$ has only two values as $s_i = s^{\pm}$. 
In the regime of phase transition of magnetics, for example, $s^+ = 1$, $s^- = -1$.
Assuming the sum of the states, $Z$, as the normalizing factor, the probability of the state is, $p (S) = e^{-\beta H(S)} / Z$. 
The sum over the state can be written as $Z_i = \sum e^{-\beta H (S_i)}$, where $S_i$ is $i$ th state out of all possible states of the system.

Normally, numerical treatment of the sum over states shows great diffuculty, simulation of the equillibrium state is usually made by 
some method of statitical sampling. For example, Monte Carlo simulation is quite classical and popular. 
In this study, Gibbs sampling will be used for the calculation.
If the $i$ th particle orientation is $s_i = s^+$ where $S' = \{s_i\}$, while the other particle orientation is given by
 $\bar{S} = {s_1, s_2, \cdots, s_{i-1}, s_{i+1}, \cdots, s_n}$, the probability of such state is given with conditional probabilyty as,
$p(S'=s^+|\bar{S}) = 1 / (1 + e^{-2 J\beta (s_{i-1} + s_{i+1})})$\cite{Imada_Miyashita_stats_mech}. 

However, such such expression of probability is assuming the particle geometry to be spherically symmertic, 
while the shape of Montmorillonite particle is rather sheet like. 
Since sheet like geometry is difficult to simulate by symmetrical geometry, we have introduced an auxualliary variable, $c$, to 
change the weight of probability where the statistical mechanics naturally assuming the law of equipartition of energy.

The variable of $i$ th patricle $s_i$ takes two values. On the other side, we assume the auxiallyary variable takes more than two values, 
$C = \{c_1, c_2, \cdots, c_n\}$, $2 < c_i \in \mathbf{N}$. By using $c_i$ as a parameter to determine $s_i$ as,
\begin{equation}
 s_i (c_i) = \left\{
  \begin{aligned}
   & s^+  (c_{\text{th}} < c_i), \\
   & s^-  (c_i \leq c_{\text{th}}), 
  \end{aligned}
       \right.
\end{equation}
where $c_{\text{th}}$ is a parameter to chenge weight of probability.

Introduction of this auxualliary valiable, we can violate low of equipartition of energy to simulate assymmetric geometry particles.
For example, assuming $0 \leq c_i \leq 10$, $c_{\text{th}} = 1$, and if $c_i$ shows statistical behaviour, 
the weight of the probability of $s_i$ shall not be symmetric; $p (s_i = s^+) = 1/10$, $p (s_i = s^-) = 9/10$, even though $p (c_i)$ shows uniform probability distribution.

By using this auxalliary variable, Hamiltonian of the system is determined by $C$,
\begin{equation}
 H (C) = - \sum J_{ij} (s_i(c_i)s_j(c_j)).
\end{equation}

Therefore, the conditional probability of $C' = c^J$ under a $\bar{C}$ state is given by,
\begin{equation}
 p (C' = c^J|\bar{C}) = \frac{1}{1+\sum_{j=0, j\not= J}e^{-\beta J \left\{(s_i (c^j) - s_i (c^J))(s_{i-1} + s_{i+1})\right\}}}. 
  \label{prob_accesory}
\end{equation}

In, eq. (\ref{prob_accesory}), interaction potential of EDL forces and van der Waals force are summerized in $J$.

To make the effetiveness of the model, we are trying to make calculations by na\''{i}ve assumption for calculating the potential as,
\begin{equation}
 J_{ij} = \frac{1}{D_{ij}} - \alpha e^{\kappa D_{ij}}.   \label{potential}
\end{equation}
Here, the first term of right side of eq. (\ref{potential}) representing van der Waals force, and the second denoting EDL forces, and $D_{ij}$ is 
distance between $i$ th and $j$ th particles.

As proposed in ref. \cite{adachi} and ref. \cite{sakairi}, the distance between particles are represented by $D = \delta (1/\phi - 1)$,
where $D$, $\delta$, and $\phi$ is distance between particle centers assuming sheet like particles array in parallel, sheet like particle width,
 and, volume fraction of the suspension. This exression can be used in the case of F-F interaction. 
When it comes to E-F interaction, we have a simplified assumtion that each particles are aligned in T-like array as shown in fig. XXX.
If we assume sheetlike particle length to be $l$, distance between particles can be represented like $\delta (1/\phi - 1) - (l-\delta)/2$.
The term $(1-\delta)/2$ reflects diminished interparticle distance of T-like aligned particles.
Of course such assumption shall be regarded as over simplification, and some adjustment coefficient that reflecting to narrow interaction area
between T-like alignment. However we suppose E-F interaction at quite lower ionic strength may cause spill-over effect as pointed in ref. \cite{sposito},
furthermore, once the overlapping area of EDL is increased a little more, the evergy of the system shall be increased.

By these assumtion, the interaction distance of neiboughing particles might be expressed as,
\begin{equation}
 D_{ij} = \delta\left(1 - \frac{1}{\phi}\right) - (s_i + s_j)\frac{l-\delta}{2}.  \label{distance}
\end{equation}
In this formula, we assumed any proper coodinate that we can put every particles in this corrdinate.
The center to center particle distance is assumed to be $D$, and they are alocated in uniformly.
Assuming, when $s_i = 0$, the particle is aligned vertical to the coordinate, and whem $s_i = 1$, the particle is aligned in horizontal to the corrdinate.
Therefore, $(s_i, s_j) = (0, 0)$ coinsides F-F alignment,  $(s_i, s_j) = (1, 0)$ or $(s_i, s_j) = (0, 1)$ coincides E-F alignment, and $(s_i, s_j) = (1, 1)$ coincides E-F alignment.

Substituting eq. (\ref{distance}) into eq. (\ref{potential}) yields,
\begin{equation}
 H (C) = \sum_{i<j} \left\{e^{-\kappa \bar{D}} - \frac{\alpha}{\bar{D}} + \kappa D_{\text{ex}}\left(s_i(c_i) + s_j(c_j)\right)\right\},
\end{equation}
where $D_{\text{ex}}$ is defined to be $(l + \delta) / 2$ and $\bar{D} = D - \delta$.
This Hamiltonian formulation leads the following conditional probebility expression at $C' = c^J$ and the rest of the system state is $\bar{C}$. 
Where $c^J$ means, the order parameter is in $J$ th state.
\begin{equation}
 p(C'=c^J|\bar{C}) = \frac{1}{1 + e^{-2\beta J \{s_i (c_i) - s_i (c^J)\}}},  \label{conditional_prob}
\end{equation}
where, $J = \alpha \kappa e^{-\kappa D}$, and, simple perturbation was given that neglects 2nd order of $D_{\text{ex}}/\bar{D}$ and $D_{\text{ex}} / \bar{D}^2$.
Also, 
\begin{equation}
 J = \alpha \kappa e^{-\kappa \bar{D}}.
\end{equation}

The form of eq. (\ref{conditional_prob}) is quite similar to that of Ising model of typical phase transition of magnetic material.
This similarity insisting the state of sheet like colloidal suspension shows first order-phase transition.





\begin{thebibliography}{99}
 \bibitem{adachi}  \textit{Trans. of JSIDRE}, \textit{No.} 200, \textit{pp.} 53-58 (1999. 4)
 \bibitem{sposito} Garrison Sposito, The Chemistry of Soils Second Edition, 
 \bibitem{sakairi} \textit{J. of Colloid and Interface Sci.}, \textit{Vol.} 283, \textit{Issue} 1, 1 March 2005, pp. 245-250
 \bibitem{Kubo}    R. Kubo, Statistical mechanics, Kyoritsu Shuppan (in Japanese)
 \bibitem{Imada_Miyashita_stats_mech} M. Imada, S. Miyashita, Univ. Tokyo Engineering Course Statitical Mechanics, Maruzen (in Japanese)
\end{thebibliography}


\end{document}