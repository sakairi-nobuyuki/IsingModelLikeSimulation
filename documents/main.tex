\documentclass{article}
\usepackage{amsmath,latexsym,bm}

\author{N. Sakairi}
\title{The origin of meta-structural strength of Montmorillonite solution using Ising model like numerical model}

\begin{document}
\maketitle

\section{Introduction}
The author had proposed the property of yield stress, a part of rheologycal behaviour of visco-elastic solution,
was assumed to be originated face-face interaction \cite{sposito}\cite{adachi}\cite{sakairi}.

In, \cite{sakairi}, rather strong repulsive force between sheet like particles of Montmorillonite was assumed.


\section{Theory}
\subsection{1 Dimension Model}
An Ising model like model was proposed to explain how each sheet like Montmorillonite particles arrayed in solution.
From the theoretical point of statistical mechanics with classical manners, equillibrium state of large number of particles 
are usually simulated by Ising model. 
This model assuming all particles in the system is located uniformly in the field with same distance, 
and, only one degree of freedom, of which values are 1 and -1.
While the interaction potential between particles are given with a coefficient $J$ and relatopnship between $i$ th and $j$ th particles.
Thus the Hamiltonial, $H$, of the system is given by, $H = -J\sum_{i<j} s_is_j$, 
where $\sum_{i<j}$ denoting summation over every possible combinations of particles.

Boltzmann factor of a state is proportional to the probability of the system is in the state.
In the subject system, repulsive force from EDL and van der Waals atractive force is of interest. 
These two forces would be dominative factors of particle alignment.
Therefore we suppose each state of the system can be represented by combinations of particle orientation.
If we assume, $s_i$ to be the orientation of the $i$ th particle in the system, the state of system can be described as a group with the representation of 
 $S = \{s_1, s_2, \cdots, s_n\}$.
For the purpose of simplicity, $s_i$ has only two values as $s_i = s^{\pm}$. 
In the regime of phase transition of magnetics, for example, $s^+ = 1$, $s^- = -1$.
Assuming the sum of the states, $Z$, as the normalizing factor, the probability of the state is, $p (S) = e^{-\beta H(S)} / Z$. 
The sum over the state can be written as $Z_i = \sum e^{-\beta H (S_i)}$, where $S_i$ is $i$ th state out of all possible states of the system.

Normally, numerical treatment of the sum over states shows great diffuculty, simulation of the equillibrium state is usually made by 
some method of statitical sampling. For example, Monte Carlo simulation is quite classical and popular. 
In this study, Gibbs sampling will be used for the calculation.
If the $i$ th particle orientation is $s_i = s^+$ where $S' = \{s_i\}$, while the other particle orientation is given by
 $\bar{S} = {s_1, s_2, \cdots, s_{i-1}, s_{i+1}, \cdots, s_n}$, the probability of such state is given with conditional probabilyty as,
$p(S'=s^+|\bar{S}) = 1 / (1 + e^{-2 J\beta (s_{i-1} + s_{i+1})})$\cite{Imada_Miyashita_stats_mech}. 
This is a very popular formula for numerical simulations of Ising model.
Trivially, the orientation of the spin is decided by spin states of neibourghing particles.

Trying to apply these quite puplular theory to the subject  problem, the form of potential shall be defined at first.
We also have assumed that only neibourghing particles affects the subject particle, and the balancing of EDL force and 
van der Waals force is dominative.
Very famous Debye-Huckel approximation, EDL force is represented by a exponetial function with a reluxation length, $\kappa$, 
called as Debye length, and, van der Waals force is proportional to the particle distance.
For simplicity, the potential between $i$ th and $j$ th particle
might be in the form of $J_{ij} = 1/D_{ij} - \alpha e^{\kappa D_{ij}}$.

As proposed in ref. \cite{adachi} and ref. \cite{sakairi}, the distance between particles are represented by $D = \delta (1/\phi - 1)$,
where $D$, $\delta$, and $\phi$ is distance between particle centers assuming sheet like particles array in parallel, sheet like particle width,
 and, volume fraction of the suspension. This exression can be used in the case of F-F interaction. 
When it comes to E-F interaction, we have a simplified assumtion that each particles are aligned in T-like array as shown in fig. XXX.
If we assume sheetlike particle length to be $l$, distance between particles can be represented like $\delta (1/\phi - 1) - (l-\delta)/2$.
The term $(l-\delta)/2$ reflects diminished interparticle distance of T-like aligned particles.
Of course such assumption shall be regarded as over simplification, and some adjustment coefficient that reflecting to narrow interaction area
between T-like alignment. However we suppose E-F interaction at quite lower ionic strength may cause spill-over effect as pointed in ref. \cite{sposito},
furthermore, once the overlapping area of EDL is increased a little more, the evergy of the system shall be increased.


By these assumtion, the interaction distance of neiboughing particles might be expressed as,
\begin{equation}
 D_\text{s} = \delta\left(1 - \frac{1}{\phi}\right) - (s_i + s_j)\frac{l-\delta}{2}.  \label{distance}
\end{equation}
In this formula, we assumed any proper coodinate that we can put every particles in this corrdinate.

Making the discussion simple, we introduce a form of perturbation method to the particle distance as,
\begin{equation}
 D_{\text{s}} = \bar{D} + \tilde{D} (s_i + s_j),
\end{equation}
\begin{equation}
 \bar{D} = D - \delta,
\end{equation}
\begin{equation}
 \tilde{D} = \frac{l - \delta}{2}.
\end{equation}
Since we assume a dilute suspension, $\tilde{D} / \bar{D}$ is sufficiently small that we can neglect its second order.

Suppose summation over neibourghing particles as $\sum_{i<j}$, Hamiltonian of the system is,
\begin{equation}
 H(S) = \sum_{i<j} \left[\frac{a_1}{\bar{D} + \tilde{D}(s_i + s_j)}  - 
		    a_2 e^{-\kappa \left\{\bar{D} + \tilde{D} (s_i + s_j)\right\}}\right].  \label{1d_H_orig}
\end{equation}

Since eq. (\ref{1d_H_orig}) seems to be complex on calculation of conditional probability, simple approximation will be introduced.
The Hamiltonian of the system is composed of van der Waals force and EDL force.
The subject system is assumed to be sufficiently dilute system, therefore van der Waals force is supposed to be smaller.
Especially, if we give a series expansion of $D$ around $\bar{D}$ with $\tilde{D}$, the terms smaller than $\mathcal{O} (\tilde{D}/\bar{D})$
might be smaller due to the characteristic of the formulation of van der Waals force to be proportional to $1/D$.
Even though the fucntion $1/D$ shows expansional behaviour around 0, $D$ is sufficiently large that we can neglect the effect.
Thus, $1/D \simeq 1/\bar{D}$.

On the other hand, EDL force is rather longer influence due to expanded EDL on lower ionic strength.
If $\tilde{D}$ is sufficiently smaller than $1/\kappa$, however, $\tilde{D}$ has on around 1nm size and $1/\kappa$ has also 1nm width.
Therefore we cannot use such na\"{i}ve perturbation for EDL force.

Following above manner, Hamiltonian is given as,
\begin{equation}
 H(S) = \sum_{i<j} \left[\frac{a_1}{\bar{D}} - a_2 e^{-\kappa \left\{\bar{D} + \tilde{D} (s_i + s_j)\right\}}\right].  \label{1d_H}
\end{equation}

Conditional probability of the subjectpartilcle when other particles' state is fixed is very important to understand 
stochastic behaviour of the system and also statistic sampling.
If we define a sub-set of the system, $S$, as $\tilde{S}$ and rest of $\tilde{S}$ as $\bar{S}$,
$S = \bar{S} + \tilde{S}$. Assuming $\tilde{S} = \{s_i\}$, where $s_i$ indicating the state of $i$ th particle,
the conditional probability of $s_i = s^+$ is $p(\tilde{S}=\{s^+\}|\bar{S}) = p(\tilde{S}=\{s^+\}, \bar{S}) / p (\bar{S})$.
By marginalizing the joint probability distribution of $p(\tilde{S}, \bar{S})$ by $\tilde{S}$, 
$p (\bar{S}) = \sum_{\tilde{S}} p(\tilde{S}, \bar{S})$. 
Now $s_i = (s^+, s^-)$, and $p (\bar{S}) = p(\tilde{S} = p^+, \bar{S}) + p(\tilde{S} = p^-, \bar{S})$.
Using Boltzmann factor, conditional probability is given as,
\begin{equation}
 p (\tilde{S} = s^+|\bar{S}) = 
  \frac{1}{1 + e^{-\beta \Delta H}}, \label{1d_cond_prob}
\end{equation}
where, 
\begin{equation}
 \Delta H = e^{-\kappa \bar{D}} \left(e^{-\kappa \tilde{D} s_{i-1}} + e^{-\kappa \tilde{D} s_{i+1}}\right)
  \left(e^{-\kappa \tilde{D} s_i^-} - e^{-\kappa \tilde{D} s_i^+}\right).  \label{1d_dH}
\end{equation}.

Eq. (\ref{1d_cond_prob}) and eq. (\ref{1d_dH}) saying that conditional probability is by states of neibourghing particles, 
and, defference between probability of the state of the particles.

When we assume, $s^- < s^+$, $p(\tilde{S} = s^-|\bar{S}) > 1/2$, and, $s^+ < s^-$, $p(\tilde{S} = s^-|\bar{S}) < 1/2$.
Also $p(\tilde{S} = s^-|\bar{S}) + p(\tilde{S} = s^+|\bar{S}) = 1$ is easily obtained, and, is obaying the definition of the probability.


\subsection{2 Dimension Model}
Next, condidering into the dimensionality of the system, 2 dimension Ising model will be considered.

As is in the 1 dimensional system, all particles are arrange uniformly in the lattice field with each particle distance is same.
Applying des Cartes co-ordinate, the position of a particle is expressed as $(i, j)$ where $i, j \in \mathbf{N}$.

Since the shape of the particle is sheet like and the system is 2 dimensional, the degree of freedom of the particle 
might be two orientation. 
As assumed in the 1 dimensional system, the value of each freedom can hold 2 values like 0 or 1.
For example, if the value is in 1, the interparticle distance is narrow, while in 0, the particle distance is wide.
These 2 degree of freedom can be expressed as $\bm{s} = (0, 1)$, $(1, 0)$, or $(1, 1)$.
Due to the geometry of the particle is sheet like, $\bm{s}$ cannot be $(0, 0)$.

By these assumntions, the state of the particle in postion of $(i, j)$ is expressed as 
$\bm{s}_{i}^{\mspace{10mu}j} = (s_{i\mspace{10mu}0}^{\mspace{10mu}j}, s_{i\mspace{10mu}1}^{\mspace{10mu}j}) \in \mathbf{N}^2$, 
where $s_{i\mspace{10mu}k}^{\mspace{10mu}j} = 0$ or $1$.
The system is expressed in the form of set as, 
$S = \{\bm{s}_{0}^{\mspace{10mu}0}, \bm{s}_{0}^{\mspace{10mu}1}, \cdots, \bm{s}_{i}^{\mspace{10mu}j},\cdots,  \bm{s}_{N_x}^{\mspace{10mu}N_y}\}$, 
where $N_x, N_y \in \mathbf{N}$

Since interaction between two particles in 2 or higher spatial dimension is defined by a function of distance, the interaction force between two particles
 shall be determined by a distance function.
Euclid distance, $d_s$, seems to be a realistic distance function. 
If we write positions of two particles as, $\bm{r}_{i}^{\mspace{10mu}j}$, and $\bm{r}_{l}^{\mspace{10mu}m}$, the distance will be 
$d_2 (\bm{r}_{i}^{\mspace{10mu}j}, \bm{r}_{l}^{\mspace{10mu}m})$. 
Implicitly, each spins are alocated on position $\bm{r}_{i}^{\mspace{10mu}j}$, and, the spin of the particle is $\bm{s}_{i}^{\mspace{10mu}j}$.
Thus, EDL interaction is assumed to be proportional to $e^{-\kappa d_2 (\bm{r}_{i}^{\mspace{10mu}j}, \bm{r}_{l}^{\mspace{10mu}m})}$.

However we are now assuming the interaction force is acting between neibourghing particles, therefore the Euclid distance is easily simplified to
 $x$-directional and $y$-directional neibourghing components as 
\begin{equation}
 d_2 (\bm{r}_{i}^{\mspace{10mu}j}, \bm{r}_{l}^{\mspace{10mu}m}) = \bar{D} - \tilde{D} \bm{e}(\bm{s}_{i}^{\mspace{10mu}j} + \bm{s}_{l}^{\mspace{10mu}m}),   \label{Euclid_dist}
\end{equation}
where, $\bm{e} = \delta^{il}\bm{e}_x +  \delta_{jm}\bm{e}_y$, and, $\bm{e}_x = (1, 0)$, $\bm{e}_y = (0, 1)$, and, $\delta_{il}$ means Kronecker's delta.

Suppose the interaction of $\bm{s}_{i}^{\mspace{10mu}j}$ shall be condidered by calculation between $\bm{s}_{i}^{\mspace{10mu}j}$ and $\bm{s}_{i-1}^{\mspace{20mu}j}$, 
$\bm{s}_{i+1}^{\mspace{20mu}j}$, $\bm{s}_{i}^{\mspace{10mu}j-1}$, and, $\bm{s}_{i}^{\mspace{10mu}j+1}$, Hatmiltonian of the system is geven by,
\begin{equation}
 H(S) = \sum_{i<l, j<m} \left[
		a_1 e^{-\kappa d_2 (\bm{r}_{i}^{\mspace{10mu}j}, \bm{r}_{j}^{\mspace{10mu}m})} 
		- \frac{a_2}{d_2 (\bm{r}_{i}^{\mspace{10mu}j}, \bm{r}_{j}^{\mspace{10mu}m})} 
	      \right].
\end{equation}

Expanding this equation with ignoring van der Waals force yielding , 
\begin{equation}
  \begin{aligned}
    H (S) = a_1 \left\{\right.&e^{-\kappa d_2 (\bm{r}_{1}^{\mspace{10mu}1}, \bm{r}_{0}^{\mspace{10mu}1})}
   + e^{-\kappa d_2 (\bm{r}_{1}^{\mspace{10mu}1}, \bm{r}_{1}^{\mspace{10mu}0})} 
   + e^{-\kappa d_2 (\bm{r}_{1}^{\mspace{10mu}1}, \bm{r}_{2}^{\mspace{10mu}1})} 
   + e^{-\kappa d_2 (\bm{r}_{1}^{\mspace{10mu}1}, \bm{r}_{1}^{\mspace{10mu}2})} \\
   &+ \cdots \\
   &+ e^{-\kappa d_2 (\bm{r}_{i}^{\mspace{10mu}j}, \bm{r}_{i-1}^{\mspace{10mu}j})} 
   + e^{-\kappa d_2 (\bm{r}_{i}^{\mspace{10mu}j}, \bm{r}_{i}^{\mspace{10mu}j-1})} 
   + e^{-\kappa d_2 (\bm{r}_{i}^{\mspace{10mu}j}, \bm{r}_{i+1}^{\mspace{10mu}j})} 
   + e^{-\kappa d_2 (\bm{r}_{i}^{\mspace{10mu}j}, \bm{r}_{i}^{\mspace{10mu}j+1})} \\
   &+ \cdots \\
   &+ e^{-\kappa d_2 (\bm{r}_{N_x}^{\mspace{10mu}N_y}, \bm{r}_{N_x-1}^{\mspace{10mu}N_y})} 
   + e^{-\kappa d_2 (\bm{r}_{N_x}^{\mspace{10mu}N_y}, \bm{r}_{N_x}^{\mspace{10mu}N_y-1})} 
   + e^{-\kappa d_2 (\bm{r}_{N_x}^{\mspace{10mu}N_y}, \bm{r}_{N_x+1}^{\mspace{10mu}N_y})} 
   + e^{-\kappa d_2 (\bm{r}_{N_x}^{\mspace{10mu}N_y}, \bm{r}_{N_x}^{\mspace{10mu}N_y+1})}\left.\right\}
  \end{aligned}
\end{equation}

Looking into a particle located on $(i, j)$, the formulation can be translated to explicitl expression by using 
eq. (\ref{Euclid_dist}),
\begin{equation}
 \begin{aligned}
 h_i & (\bm{r}_{i}^{\mspace{10mu}j}, \bm{r}_{i-1}^{\mspace{10mu}j}, \bm{r}_{i+1}^{\mspace{10mu}j}, \bm{r}_{i}^{\mspace{10mu}j-1}, \bm{r}_{i}^{\mspace{10mu}j+1})\\
  & = e^{-\kappa d_2 (\bm{r}_{i}^{\mspace{10mu}j}, \bm{r}_{i-1}^{\mspace{10mu}j})} 
   + e^{-\kappa d_2 (\bm{r}_{i}^{\mspace{10mu}j}, \bm{r}_{i}^{\mspace{10mu}j-1})} 
   + e^{-\kappa d_2 (\bm{r}_{i}^{\mspace{10mu}j}, \bm{r}_{i+1}^{\mspace{10mu}j})} 
   + e^{-\kappa d_2 (\bm{r}_{i}^{\mspace{10mu}j}, \bm{r}_{i}^{\mspace{10mu}j+1})}  \\
  & = 
  e^{-\kappa \bar{D}}\left\{
  e^{-\kappa \tilde{D}\bm{e}_y\cdot(\bm{s}_{i}^{\mspace{10mu}j} + \bm{s}_{i-1}^{\mspace{10mu}j})}
  + e^{-\kappa \tilde{D}\bm{e}_y\cdot(\bm{s}_{i}^{\mspace{10mu}j} + \bm{s}_{i+1}^{\mspace{10mu}j})}
  + e^{-\kappa \tilde{D}\bm{e}_x\cdot(\bm{s}_{i}^{\mspace{10mu}j} + \bm{s}_{i}^{\mspace{10mu}j-1})}
  + e^{-\kappa \tilde{D}\bm{e}_x\cdot(\bm{s}_{i}^{\mspace{10mu}j} + \bm{s}_{i}^{\mspace{10mu}j+1})}\right\} \\
  & = 
  e^{-\kappa \bar{D}}\left\{
  e^{-\kappa \tilde{D}\bm{e}_y\cdot\bm{s}_{i}^{\mspace{10mu}j}}
  \left(
  e^{-\kappa \tilde{D}\bm{e}_y\cdot\bm{s}_{i-1}^{\mspace{10mu}j}}
  + e^{-\kappa \tilde{D}\bm{e}_y\cdot\bm{s}_{i+1}^{\mspace{10mu}j}}
  \right)
  + e^{-\kappa \tilde{D}\bm{e}_x\cdot\bm{s}_{i}^{\mspace{10mu}j}}
  \left(
  e^{-\kappa \tilde{D}\bm{e}_x\cdot\bm{s}_{i}^{\mspace{10mu}j-1}}
  + e^{-\kappa \tilde{D}\bm{e}_x\cdot\bm{s}_{i}^{\mspace{10mu}j+1}}
  \right)
  \right\}  \label{part_H}
 \end{aligned}
\end{equation}

Since the particle state is defined by sheet like particle orientation as $(1, 0)$, $(0, 1)$, or, $(1, 1)$, 
the distance term can be simplified by putting the exponential function of $\bm{s}$ to be out of the function as 
$e^{\kappa\tilde{D}\bm{e}_y\cdot\bm{s}_i^{\mspace{5mu}j}} = 1-(1-e^{-\kappa\tilde{D}})\bm{e}_y\cdot\bm{s}_i^{\mspace{5mu}j}$.
This yields,
\begin{equation}
\begin{aligned}
 h_i & (\bm{r}_{i}^{\mspace{10mu}j}, \bm{r}_{i-1}^{\mspace{10mu}j}, \bm{r}_{i+1}^{\mspace{10mu}j}, \bm{r}_{i}^{\mspace{10mu}j-1}, \bm{r}_{i}^{\mspace{10mu}j+1})\\
 = & e^{-\kappa\bar{D}}
  \left[
   (1 - \Upsilon\bm{e}_y\cdot\bm{s}_i^{\mspace{5mu}j})
   \left\{2 - \Upsilon\bm{e}_y\cdot(\bm{s}_{i-1}^{\mspace{5mu}j} + \bm{s}_{i+1}^{\mspace{5mu}j})\right\}\right. \\
   & \left.+ (1 - \Upsilon\bm{e}_x\cdot\bm{s}_i^{\mspace{5mu}j})
   \left\{2 - \Upsilon\bm{e}_x\cdot(\bm{s}_{i}^{\mspace{5mu}j-1} + \bm{s}_{i}^{\mspace{5mu}j+1})\right\}
  \right],
\end{aligned}
\end{equation}
where, $\Upsilon = 1-e^{\kappa\tilde{D}}$.

As is same manner as 1 dimensional case, the sub-set with only one element is denoted as $S' = \{\bm{s}_{ij}\}$, and, 
the complementary set of is denoted as $\bar{S} = S - S'$.

Assuming when the particle of $S'$ is in a fixed state as $u$, $S' = \bm{s}_{i\mspace{7mu}u}^{\mspace{5mu}j}$. 
As an example, $\bm{s}_{i\mspace{7mu}1}^{\mspace{5mu}j} = (1, 0)$,
 $\bm{s}_{i\mspace{7mu}2}^{\mspace{5mu}j} = (1, 0)$, or, $\bm{s}_{i\mspace{7mu}3}^{\mspace{5mu}j} = (1, 1)$.

Trying to obtain conditional probability of $p(S'=\{\bm{s}_{i\mspace{7mu}u_{\xi}}^{\mspace{5mu}j}\}|\bar{S})$, the total number of the freedom of the particle is 3,
and, the probability shall be 
$p(S'=\{\bm{s}_{i\mspace{7mu}u_{\xi}}^{\mspace{5mu}j}\}|\bar{S}) = p(S'=\{\bm{s}_{i\mspace{7mu}u_{\xi}}^{\mspace{5mu}j}\}, \bar{S})/\sum_{\mu} p(S'=\{\bm{s}_{i\mspace{7mu}u_{\mu}}^{\mspace{5mu}j}\}, \bar{S})$
This will easily lead to,
\begin{equation}
 p(S'=\{\bm{s}_{i\mspace{7mu}u_{\xi}}^{\mspace{5mu}j}\}|\bar{S}) = 
  \frac{1}{1
  + \frac{p(S'=\{\bm{s}_{i\mspace{7mu}u_{\eta}}^{\mspace{5mu}j}\}, \bar{S})}{p(S'=\{\bm{s}_{i\mspace{7mu}u_{\xi}}^{\mspace{5mu}j}\}, \bar{S})}
  + \frac{p(S'=\{\bm{s}_{i\mspace{7mu}u_{\zeta}}^{\mspace{5mu}j}\}, \bar{S})}{p(S'=\{\bm{s}_{i\mspace{7mu}u_{\xi}}^{\mspace{5mu}j}\}, \bar{S})}}.
\end{equation}

The term of divergence of each joint probability is,
\begin{equation}
 \frac{p(S'=\{\bm{s}_{i\mspace{7mu}u_{\eta}}^{\mspace{5mu}j}\}, \bar{S})}{p(S'=\{\bm{s}_{i\mspace{7mu}u_{\xi}}^{\mspace{5mu}j}\}, \bar{S})}
  = e^{-\beta\left\{H\left(S'=\{\bm{s}_{i\mspace{7mu}u_{\eta}}^{\mspace{5mu}j}\}, \bar{S}\right)
	     - H\left(S'=\{\bm{s}_{i\mspace{7mu}u_{\eta}}^{\mspace{5mu}j}\}, \bar{S}\right)\right\}}.
\end{equation}

The difference of Hamiltonian will be,
\begin{equation}
 \begin{aligned}
    H(S'=\{\bm{s}_{i\mspace{7mu}u_{\eta}}^{\mspace{5mu}j}\}, \bar{S})
  - &H(S'=\{\bm{s}_{i\mspace{7mu}u_{\xi}}^{\mspace{5mu}j}\}, \bar{S}) \\
   = e^{-\kappa\bar{D}}\Upsilon(\bm{s}_{i\mspace{5mu}u_{\eta}}^{\mspace{5mu}j} - \bm{s}_{i\mspace{5mu}u_{\xi}}^{\mspace{5mu}j})\cdot
  &\left[\bm{e}_y\{2 - \Upsilon\bm{e}_y\cdot(\bm{s}_{i-1}^{\mspace{5mu}j} + \bm{s}_{i+1}^{\mspace{5mu}j})\}\right.\\
  &\left.+ \bm{e}_x\{2 - \Upsilon\bm{e}_x\cdot(\bm{s}_{i}^{\mspace{5mu}j-1} + \bm{s}_{i}^{\mspace{5mu}j+1})\}\right]
 \end{aligned}
\end{equation}



\begin{thebibliography}{99}
 \bibitem{adachi}  \textit{Trans. of JSIDRE}, \textit{No.} 200, \textit{pp.} 53-58 (1999. 4)
 \bibitem{sposito} Garrison Sposito, The Chemistry of Soils Second Edition, 
 \bibitem{sakairi} \textit{J. of Colloid and Interface Sci.}, \textit{Vol.} 283, \textit{Issue} 1, 1 March 2005, pp. 245-250
 \bibitem{Hiemenz} Paul C. Hiemenz, and, Raj Rajagopalan, Principles of Colloid and Surface Chemistry, Revised and Expanded (UNDERGRADUATE CHEMISTRY SERIES), CRC Press
 \bibitem{Kubo}    R. Kubo, Statistical mechanics, Kyoritsu Shuppan (in Japanese)
 \bibitem{Imada_Miyashita_stats_mech} M. Imada, S. Miyashita, Univ. Tokyo Engineering Course Statitical Mechanics, Maruzen (in Japanese)
\end{thebibliography}


\end{document}